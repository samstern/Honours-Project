\chapter{Data}


\section{Overview of the HES Dataset}
The data used in this project comes from The Household Electricity Survey (HES), which was a survey undertaken by the DEFRA to monitor the electrical power demand and energy consumption of individual households in England over the period May 2010 to July 2011 \cite{HES}. The aim of the study was to identify and catalogue the range and quantity of electrically powered appliances found in a typical home, understand the household's frequency and patterns of electricity usage and to collect `user habit' data that emerge from using a range of appliances \cite{early findings}.

The HES study monitored 250 households, of which 26 were monitored for one year while the remaining 224 were monitored for roughly one month. Each household had between 13 and 85 individual appliances being monitored in their homes such that when aggregated (as outlined in section \aggregationSection), the result gives an estimate of a mains reading. Depending on the household, measurements were either taken in 2 or 10 minute intervals with units of kilowatt hours (kWh).

In addition to data regarding the appliance types and data readings, participating households also kept diaries of how they used their main appliances and provided information about the household such as the number of occupants, employment status, IPSOS social-grade and whether there are children present in the household.


\section{Extracting the Data and Pre-Processing}

As explained in section \mentionOfAggregationSection, electricity readings of individual appliances and sockets were taken for each household, as opposed to the total energy used as was required for this project. The HES study recorded measurements for 250 possible appliances that a household could have (giving values of 0 to appliances that weren't monitored). The resulting raw data was large csv files with largely redundant entries. 

The first step in pre-processing the data was to use create a MySQL database and import the the appliance readings into a table. Cambridge Architectural Research Ltd had additional files that mapped which appliances needed to be aggregated for each household in order to create an estimate for the mains reading, this was often not simply the sum of all appliances readings. A table was therefore created for every household where each row contained the aggregated electricity measurements for a given date and time. 

250 households participated in the HES study, which is a relatively small number for a machine learning task as there might not be enough data to build models that accurately sample the entire English population. To help account for this, the 26 households that were monitored for an entire year were split into 12 instances that could be treated as separate households, resulting in an additional 281 household instances. While this does not create a more diverse group, it does add more instances to train, validate and test a classifier with.

Next, the inconsistency in measurement intervals was accounted for. While some households reported how much energy they used in 10 minute intervals, others were measure in 2 minute intervals. To create consistency in the data, for the `2-minute households', every five intervals were summed so that all households had 10 minute granularity. This step is important since some consumption features, would be affected by a difference differences in measurement intervals.

The last stage in pre-processing was to ensure that each instance was of the same length. As explained in \fourierSection, temporal structure was observed both intraday and intraweek. Therefore, the timeseries instances were manipulated so that they each had a length of 28 days and started on the same day of the week.





\section{Issues}
\begin{enumerate}
\item Homes were not perfectly representative of the population
\begin{itemize}
	\item only homeowners were included
	\item only concidered homes in England, not the entire UK
	\item class size ratios not representative of population
\end{itemize} 
\item 
\item The purpose of the project was to determine whether it is \textit{possible} to distinguish between households, and to show how this might be achieved.  
\item Several households have periods where their energy consumption pattern vanishes and very little or no energy is used. It is likely that these are periods where the members of the household are away or on holiday.
\item The `total' electricity is not always well estimated. 
\item Initially, data from the IDEAL study was going to be used however as this was not available, data from the HES study was used. This resulted in a delay to the project.
\end{enumerate}

\section{Comparison to Previous Work}







